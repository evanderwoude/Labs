\documentclass[12pt]{scrartcl}


\setlength{\parindent}{0pt}
\setlength{\parskip}{.25cm}

\usepackage{graphicx}

\usepackage{xcolor}

\definecolor{darkred}{rgb}{0.5,0,0}
\definecolor{darkgreen}{rgb}{0,0.5,0}
\usepackage{hyperref}
\hypersetup{
  letterpaper,
  colorlinks,
  linkcolor=red,
  citecolor=darkgreen,
  menucolor=darkred,
  urlcolor=blue,
  pdfpagemode=none,
  pdftitle={Introduction To Git},
  pdfauthor={Christopher M. Bourke},
  pdfcreator={$ $Id: cv-us.tex,v 1.28 2009/01/01 00:00:00 cbourke Exp $ $},
  pdfsubject={PhD Thesis},
  pdfkeywords={}
}

\definecolor{MyDarkBlue}{rgb}{0,0.08,0.45}
\definecolor{MyDarkRed}{rgb}{0.45,0.08,0}
\definecolor{MyDarkGreen}{rgb}{0.08,0.45,0.08}

\definecolor{mintedBackground}{rgb}{0.95,0.95,0.95}
\definecolor{mintedInlineBackground}{rgb}{.90,.90,1}

%\usepackage{newfloat}
\usepackage[newfloat=true]{minted}
\setminted{mathescape,
               linenos,
               autogobble,
               frame=none,
               framesep=2mm,
               framerule=0.4pt,
               %label=foo,
               xleftmargin=2em,
               xrightmargin=0em,
               startinline=true,  %PHP only, allow it to omit the PHP Tags *** with this option, variables using dollar sign in comments are treated as latex math
               numbersep=10pt, %gap between line numbers and start of line
               style=default, %syntax highlighting style, default is "default"
               			    %gallery: http://help.farbox.com/pygments.html
			    	    %list available: pygmentize -L styles
               bgcolor=mintedBackground} %prevents breaking across pages
               
\setmintedinline{bgcolor={mintedBackground}}
\setminted[text]{bgcolor={mintedBackground},linenos=false,autogobble,xleftmargin=1em}
%\setminted[php]{bgcolor=mintedBackgroundPHP} %startinline=True}
\SetupFloatingEnvironment{listing}{name=Code Sample}
\SetupFloatingEnvironment{listing}{listname=List of Code Samples}


\title{CSCE 155 - C}
\subtitle{Lab 08 - Debugging}
\author{Dr.\ Chris Bourke}
\date{~}

\begin{document}

\maketitle

\section*{Prior to Lab}

Before attending this lab:
\begin{enumerate}
  \item Read and familiarize yourself with this handout.
  \item Review the following free textbook resources:
	\begin{itemize}
  	  \item GDB Tutorial \url{http://web.eecs.umich.edu/~sugih/pointers/summary.html} 
	  \item GNU's GDB Tutorial \url{http://dirac.org/linux/gdb/}
	\end{itemize}
\end{enumerate}

\section*{Peer Programming Pair-Up}

To encourage collaboration and a team environment, labs will be
structured in a \emph{pair programming} setup.  At the start of
each lab, you will be randomly paired up with another student 
(conflicts such as absences will be dealt with by the lab instructor).
One of you will be designated the \emph{driver} and the other
the \emph{navigator}.  

The navigator will be responsible for reading the instructions and
telling the driver what to do next.  The driver will be in charge of the
keyboard and workstation.  Both driver and navigator are responsible
for suggesting fixes and solutions together.  Neither the navigator
nor the driver is ``in charge.''  Beyond your immediate pairing, you
are encouraged to help and interact and with other pairs in the lab.

Each week you should alternate: if you were a driver last week, 
be a navigator next, etc.  Resolve any issues (you were both drivers
last week) within your pair.  Ask the lab instructor to resolve issues
only when you cannot come to a consensus.  

Because of the peer programming setup of labs, it is absolutely 
essential that you complete any pre-lab activities and familiarize
yourself with the handouts prior to coming to lab.  Failure to do
so will negatively impact your ability to collaborate and work with 
others which may mean that you will not be able to complete the
lab.  

\section{Lab Objectives \& Topics}
At the end of this lab you should be familiar with the following
\begin{itemize}
  \item Be able to better understand the errors generated by the GCC compiler
  \item Determine the types of errors that occur in a program
  \item Debug a program using GDB (the GNU Project Debugger)
\end{itemize}

\section{Background}

During these first few weeks of classes, you've likely had an error in 
one of your programs.  The error might have been something that 
made your program produce incorrect output, crash while it was 
running, or even fail to compile at all.  Unfortunately, errors like these 
are common, and it's a rare (and fantastic) moment when code gets 
written the first time without any errors.  Fortunately, because errors 
are so common, there are a myriad of tools designed to ease the 
pain of correcting them.  Errors tend to fall into one of three categories: 
compilation errors, runtime errors, and logic errors.  

\subsection*{Compilation Errors}

Compilation errors occur when the compiler (GCC in our case) 
encounters something that it doesn't understand in the code you're 
attempting to compile.  This will cause the compiler to stop and print 
a message to the screen describing the problem that it encountered.  
An example might look something like:

\begin{minted}{text}
Lab08.c:5:8: error: 'carCost' undeclared (first use in this function)
Lab08.c:5:8: note: each undeclared identifier is reported only 
	once for each function it appears in
\end{minted}

These messages can sometimes be cryptic, but they give the 
information needed to find and correct the error. There are several 
different sections in the error, each separated by a colon.

\begin{itemize}
  \item \mintinline{text}{Lab08.c}: The file that the error was encountered in.  
  	This is crucial in larger programs that use several different source and 
	header files to keep the code organized.  
  \item \mintinline{text}{5}: The first number after the file name contains 
	the line number in the file that the error can be found on.  So now we 
	know that if we go to \mintinline{text}{Lab08.c}, we'll find the error on 
	(or around) the fifth line.  

  \item \mintinline{text}{8}: The column that the error is found in (i.e., how 
	many characters (including spaces) precede the error on the line). 

  \item \mintinline{text}{error/warning/note}: The type of problem encountered.  
	Errors will stop the code from compiling.  Warnings will compile, but still 
	need to be fixed since the warning is likely trying to prevent you from 
	doing something that is legal as far as C is concerned, but may not 
	produce the desired result.  Notes are messages from the compiler 
	regarding the previous warning or error message.

  \item The rest: A brief description of the cause of the problem.  In this case, 
	the error is that the variable was undeclared.  GCC encountered this 
	variable for the first time on line 5, but it doesn't have a type to identify 
	it.  The note is to let you know that there may be other uses of 
	\mintinline{c}{carCost} in the same function, but to cut down on tons of 
	output, only the first time it was used undeclared is noted.   
\end{itemize}
	
Note that the error may not be on exactly the line number given, but will still 
be associated with the error in some way.  For example, compiling the 
following code:

\begin{minted}{c}
int main(int argc, char **argv) {
  int catAge
  catAge = 6;
  return 0;
}
\end{minted}

will cause GCC to output this error message:

\begin{minted}{text}
cat.c: In function 'main':
  cat.c:3:8: error: expected '=', ',', ';', 'asm' or '__attribute__' 
	before 'catAge'
  cat.c:3:8: error: 'catAge' undeclared (first use in this function)
  cat.c:3:8: note: each undeclared identifier is reported only 
	once for each function it appears in
\end{minted}

The error is supposedly occurring on line 4, but \mintinline{c}{catAge = 6;} 
is a valid and error free C statement.  However, the description says that 
it was expecting one of those symbols \mintinline{text}{'=', ',',} etc. 
\emph{before} \mintinline{c}{catAge}.  The thing preceding \mintinline{c}{catAge} 
is the declaration of \mintinline{c}{catAge}, which is missing the expected 
\mintinline{text}{';'} mentioned in the first error.  Fixing this also fixes the 
second error mentioned because the variable \mintinline{c}{catAge} 
has now been correctly declared.  

This also applies to errors with unbalanced brackets, \mintinline{c}{{ }} and 
parentheses \mintinline{c}{( )}.  It may find the first of the pair but be missing 
the last one, which will result in the compiler assuming that there's a problem 
on the last line of the file.  Be careful in this case, because that may not (and 
likely isn't) the place where the missing parenthesis or curly brace should go.  
Proper indentation helps immensely when trying to find the missing 
parenthesis/brace.

In most editors, the line number is not displayed by default.  The following is 
a brief description regarding how to show the line numbers in the most common 
editors used in this class.

\begin{itemize}
  \item Vim/vi: Type \mintinline{text}{:set number} while in normal mode.  
  \item Emacs: \mintinline{text}{M-x linum-mode}.  To enable this on startup for all file types, 
	place \mintinline{text}{global-linum-mode 1} in your \mintinline{text}{.emacs}
	file in your home directory.
  \item Nano/Pico: Line numbers can't be displayed along the margins as they would 
	in Notepad++ , Vim, or Emacs, but you can view the current line that the cursor 
	is placed on by pressing CTRL+C.  
  \item Notepad++:  Should be enabled by default.
\end{itemize}

\subsection*{Runtime Errors}

Runtime errors are errors that cause the program to crash (i.e., end in an 
unexpected place and manner).  Depending on the operating system you're 
running the program on, a variety of error messages might be displayed.  
Dividing by zero in C is unhandled: it might cause the program to stop, or 
it could keep running as if everything were fine, but it's still an example of 
a runtime error.  

A common runtime error is a segmentation fault (segfault).  This occurs in 
several situations, such as when the program tries to write/access memory 
that it doesn't own or when a buffer overflow occurs (which typically happens 
when you try to write to memory outside the bounds of an array, for example).  

Forgetting to place an ampersand in front of the variable in a \mintinline{c}{scanf} 
typically also causes segmentation faults.  The program will try to write the 
scanned information to the address in memory represented by the number 
stored in the variable being scanned to, rather than the address of the variable 
itself.  

The compiler will not catch these errors, but this lab will give you a couple of 
methods to help determine where and why the error occurred.  

\subsection*{Logic Errors}

Logic errors are errors that cause the program to operate incorrectly (such 
as producing incorrect output) but will compile and run without error.  An 
example of such an error would be writing a function \mintinline{c}{add} 
in the following manner:

\begin{minted}{c}
int add(int num1, int num2) {
  return num1 * num2;
}
\end{minted}

The code above would compile and execute without error, but obviously 
when you call a function named \mintinline{c}{add}, you expect it to return the sum of 
the numbers and not the product.  

\section{Activities}

Clone the code for this lab from GitHub using the following URL: 
\url{https://github.com/cbourke/CSCE155-C-Lab08}.  The project 
contains a version of Conway's Game of Life, but contains errors.
The basic rules of this game can be found here: 
\url{http://en.wikipedia.org/wiki/Conway%27s_Game_of_Life}.

\subsection{Correcting Compilation Errors}

\begin{enumerate}
  \item Try compiling the program using the supplied makefile (type 
	\mintinline{text}{make} in the same directory as the makefile).  
	Several errors will be produced.  Correct each of these compilation 
	errors.
  \item Fill out question 1 on the worksheet.
\end{enumerate}

\subsection{Correcting Runtime Errors}

\begin{enumerate}
  \item Try running the program created in the previous step 
  	(\mintinline{text}{./gameOfLife.x}).  The program should crash due 
	to a segmentation fault.  The following steps will show you how 
	to quickly find and correct segfaults using gdb.  
  \item Close the currently running \mintinline{text}{gameOfLife} program 
	by using Control-C.
  \item Type gdb on the command line.  GDB is typically used in 
  	conjunction with the \mintinline{text}{-g} flag in gcc.  The flag 
	produces debugging symbols that GDB can use to aid in the 
	debugging process (specifically, it allows for things like showing 
	which line of code the error occurred on and retaining variable names 
	in the debugger).  The \mintinline{text}{-g} flag has already been added 
	to the compilation command in the makefile, so nothing extra needs to be done.
  \item Now we have to tell GDB which file we want to debug (i.e., the program 
	we generated).  On the GDB command prompt, type \mintinline{text}{file gameOfLife.x}.  
	You should see output that looks something like \mintinline{text}{Reading symbols ... done}
  \item Now run the code from within GDB by typing run.  The program will 
	crash in the same place that it crashed before, only now we can tell exactly 
	what line of code caused it to crash.  Type \mintinline{text}{backtrace} at the GDB prompt 
	to see the list of functions on the call stack (that is, all the functions that 
	were called immediately preceding the crash).  For this first error, your 
	output should look like:

\begin{minted}{text}
#0  0x0000000100001056 in placeBeacon (gameBoard=0x0, cell=...) 
	at gameOfLife.c:60
#1  0x0000000100000f7c in main (argc=1, argv=0x7fff5fbff318) at 
	gameOfLife.c:21
\end{minted}

	The bottom function (\mintinline{c}{main}) is the first function that was called.  
	\mintinline{c}{placeBeacon} is the second function, and also where the crash 
	occurred (line 60).  However, this is a bit deceiving.  Notice that the first 
	argument to \mintinline{c}{placeBeacon} is the address \mintinline{text}{0x0}, which corresponds
	to \mintinline{c}{NULL}.  On line 60, this address is dereferenced, but clearly 
	something happened before \mintinline{c}{placeBeacon} that caused the 
	\mintinline{c}{gameBoard} to be \mintinline{c}{NULL}.
  \item To find out where the error originated from, we'll step through the program 
	line by line.  First, we'll tell GDB to insert a breakpoint somewhere, which 
	will temporarily pause execution of the code.  Type \mintinline{text}{break main} to tell 
	GDB to stop execution as soon as the \mintinline{c}{main} function is called.
  \item Now type \mintinline{text}{run} again, and restart the program.  You can use the commands 
	\mintinline{text}{next} and \mintinline{text}{step} to continue execution 
	line by line.  The \mintinline{text}{next} command will execute entire functions 
	(i.e., it won't execute the function line by line), while 
	\mintinline{text}{step} will ``step into'' a function, and execute each line of that 
	function one at a time.  Use \mintinline{text}{next} until you arrive at line 31, 
	and view the contents of the board variable by typing \mintinline{text}{print board}.  Notice 
	that even after \mintinline{text}{createGameBoard} was executed, \mintinline{text}{board} 
	remained \mintinline{text}{NULL}.  Now we've can narrow the problem down 
	to the function \mintinline{text}{createGameBoard}. 
  %\item Exit GDB by typing \mintinline{text}{quit} and look at \mintinline{c}{createGameBoard} 
	%in \mintinline{text}{gameFunctions.c} and examine the return value of the 
	%function (or the lack thereof) .  Correct the error by returning the newly created 
	%\mintinline{c}{gameBoard} variable from \mintinline{c}{createGameBoard}.
  \item There are still several runtime errors to correct.  Many of them will be identified 
	slightly more directly by GDB (it will report the line number a segfault occurs on), 
	while one or two will require you to step through the program while examining 
	variables and their addresses.
  \item Reopen GDB, type \mintinline{text}{make} at the GDB command prompt, and 
	rerun the program with \mintinline{text}{run}.  Once again, type \mintinline{text}{backtrace}, 
	identify where the program crashed, fix the error, and continue on until the 
	program no longer crashes (hint: common reasons for segfaults are \mintinline{text}{NULL} 
	pointer errors,  out-of-bounds array access, and accessing previously freed memory).
\end{enumerate}

\subsection{Correcting Logic Errors}

\begin{enumerate}
  \item There are still several errors in the program, though none will be as 
  	obvious as compilation or runtime errors.  When corrected, the program 
	should look something like Figure \ref{fig:gameOfLife}.
\begin{figure}
\centering
\includegraphics[scale=1.0]{gameOfLife}
\caption{Game of Life Depiction}
\label{fig:gameOfLife}
\end{figure}
	The two shapes on the left will oscillate but not move, while the shape 
	towards the bottom will move down and to the right.
  \item Using the rules found on the Wikipedia page for the game (linked above) to 
  	fix the program.  It's up to you whether or not to use GDB to help you find these 
	errors (most are isolated to a single function).
  \item Answer questions 2 and 3 on the worksheet.  
\end{enumerate}

\section{Advanced Activity (Optional)}

Experiment with GDB.  There are tons of various commands you can use 
to help you debug your code.  Take a look at \url{http://sourceware.org/gdb/onlinedocs/gdb/index.html#Top} 
(specifically the running, stopping, reverse execution, data, stack, and 
source links).  

Emacs is an editor with some fairly advanced built in functionality.  While there 
is a bit of a learning curve to Emacs (and Vim), the effort required to learn it will 
pay off in the long run.  Try using Emacs as an editor for a while until you're 
comfortable with some of the basic commands.  Use this website to help get 
you started.  A cheat sheet, such as this one can be extremely helpful for a 
quick reference.

Now that you can move around Emacs a bit more comfortably, open today's 
files and try using some of the commands outlined here to use GDB within 
Emacs.  Notice how the line of code currently being executed is displayed 
within the editor itself.  This can be extremely handy when trying to debug 
a program.

Alternatively, most IDEs have a debugging facility built in.  You can try redoing 
this lab in CodeBlocks using GDB through the CodeBlocks GUI.  

\end{document}
